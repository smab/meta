
\subsection{Overview}
The configuration interface can be reach as a webpage on
\url{http://localhost:<config_port>/config} where the default port is 8081.

There are six source files directly associated with this module, which can be
found in the relevant directories:

  \begin{description}
    \item[config.py]
     the main module source code, responsible for handling all requests in the configuration interface
    \item[config\_base.html]
     base template file
    \item[config\_setup.html]
     template file for Setup
    \item[config\_game.html]
     template file for Settings
    \item[config\_bridges.html]
     template file for Bridges
  \end{description}

\subsection{Configuration file}
Everything under Setup and Settings can also be changed manually in
\texttt{config.json}.

\subsubsection{Attributes}
  \begin{description}
    \item[\texttt{game\_name}:] name of initial game module to start (default: \texttt{"default"})
    \item[\texttt{game\_path}:] list of paths to where games can be found (default: \texttt{["src/games"]})
    \item[\texttt{lampdest}:] address to lamp server (default: \texttt{"localhost"})
    \item[\texttt{lampport}:] port to connect to lamp server (default: \texttt{4711})
    \item[\texttt{serverport}:] port the game server will listen on (default: \texttt{8080})
    \item[\texttt{configport}:] port the config server will listen on (default: \texttt{8081})
    \item[\texttt{stream\_embedcode}:] HTML string with stream embed code (default: \texttt{""})
    \item[\texttt{light\_ssl}:] set to \texttt{true} to connect to the lamp server with HTTPS
    \item[\texttt{light\_certfile}:] only needed if \texttt{light\_ssl} is \texttt{true}
    \item[\texttt{light\_pwd}:] password used to authorize with the lamp server
    \item[\texttt{config\_ssl}:] when \texttt{true}, run config interface on a HTTPS server
    \item[\texttt{config\_certfile}:] only needed if \texttt{config\_ssl} is \texttt{true}
    \item[\texttt{config\_keyfile}:] only needed if \texttt{config\_ssl} is \texttt{true}
    \item[\texttt{config\_pwd}:] when set, password needed to access config
  \end{description}

\subsection{Configuration interface}
\subsubsection{Setup}

This page contains settings for connections and external services.
Some of settings may require that the server is restarted.

\subsubsection{Connections}

  \begin{description}

    \item[Status]
        Show the current connection status to the lamp server.
    \item[Lamp server]
        Address to the lamp server and what port to connect to. (default:
        \texttt{localhost:4711})
    \item[Game server port]
        This is the port the web server that serves the game interface will
        listen on. (default: \texttt{8080})
    \item[Config server port]
        This is the port the web server that serves the configuration
        interface will listen on. (default: \texttt{8081})

  \end{description}

  Note: Do not use the same port for game and config or the game interface
  will not be able to start.

\subsection{Streaming}
  \begin{description}
    \item[Embed code]
        The content in this text area will be inserted in the game interface
        and is primarily for embedding a live stream player. (default:
        \textit{empty})
  \end{description}


\subsubsection{Settings}
This pages is used to change game and game settings.

\subsubsection{Changing game}
All games found in the game paths is listed to the left and the current game
can be changed directly by clicking on any of the listed games.  Note:
Changing game paths has to be done by editing \texttt{game\_paths} in the
config file.

\subsubsection{Game settings}
Game settings are shown on the right and will show the config template
specified by \texttt{config\_file} in the the running game module. For more
information, see the documentation on game settings for each module.


\subsection{Bridges}
This menu handles all the Hue bridges that are used. It lists all bridges the
lightserver knows about, sorted by their MAC-address.

\subsubsection{Identifying Selected}
By selecting bridges with the checkbox, you can tell the bridge to blink all
the lamps associated with this bridge by pressing the "Identify Selected"
button.

\subsubsection{Valid username}
Each Hue bridge needs a username. This acts like a password so unauthorized
people cannot send commands to your lamps. Each bridge in the table has a
property \texttt{valid\_username} which tells you if the lightserver knows about a
valid username for that bridge. By clicking the property, you can tell the
lightserver to generate a new username. A confirmation box will appear and you
will have to physically  press the link button on the bridge before accepting.

\subsubsection{Adding bridges}
You can add a new bridge by writing its IP address in the textfield and
pressing ``Add bridge.''

\subsubsection{Removing bridges}
You can remove bridges by select them with the checkboxes and then pressing
``Remove Selected.''

\subsubsection{Refresh List}
A cached copy of the bridges will be saved on the webserver to decrease
waiting time. If any changes happens to the bridges it will not be shown
however. You can force a request for a new version by pressing the ``Refresh
List'' button.

\subsubsection{Search Bridges}
If you know there are bridges on the same local network as the lightserver,
you can have it search for them. It will do that by a broadcast message and by
reading the Philips website. This search may take some time though and after
about 20s you \textit{must} refresh the list by pressing the ``Refresh List''
button.


\subsection{Grid}

This page is used to configure the lamp positions in the grid and the
dimensions of the grid.  The grid overview shows the positions where lamps are
placed and the status of those lamps.

\subsubsection{Color coding (interface)}
  \begin{description}
    \item[white] no lamp placed
    \item[blue]  a lamp is placed here and no errors found (note: the lamp
                 server can't know if a lamp is reachable or not)
    \item[red]   a lamp is placed here but there is no bridge with this lamp id
  \end{description}

\subsubsection{Color coding (lamps)}
  \begin{description}
    \item[white] activated lamp
    \item[blue]  placed lamp
    \item[red]   removed lamp
  \end{description}

\subsubsection{Placing lamps}
Before placing the lamps, it is recommended to first turn off any running
games. It can be done directly from the grid page by clicking on the ``Turn
Off'' button.

When you open the page, a lamp light up and then can you say which lamp was
lit. This can be done either by clicking on the overview or manually enter the
position and then click on ``Place Lamp''. Attempting to place a lamp on an
occupied position will instead remove the currently placed lamp on that
position and queue it for placement.

\subsubsection{Changing size}
Changing size is done by entering the dimensions as \texttt{widthxheight} and then
click ``Change Size''.  Note: This will clear the grid and because of that the
``Change Size''-button will be disabled to prevent loss of unsaved changes.
Saving or refreshing will reenable the button.

\subsubsection{Actions}
  \begin{description}
    \item[Save]             Send and use the grid on the lamp server.
    \item[Refresh]          Discard the local grid and request the one currently used on the lamp server.
    \item[Clear]            Clear the grid from placed lamps.
    \item[Turn Off]         Cancel any running game and turn off all lamps.
    \item[Run Diagnostics]  Run diagnosics to test the grid on the lamp server.
  \end{description}

\subsubsection{Errors}
  \begin{description}
    \item[No activated lamp]  there is either no more lamps available that isn't already placed
    \item[Invalid position]   the position were in an incorrect format and couldn't be parsed be the server
    \item[Invalid size]       the size were in an incorrect format and couldn't be parsed be the server
    \item[Invalid lamp]       the placed lamp is no longer valid and can't be placed
    \item[Saving failed]      given when the lamp server couldn't save the grid
  \end{description}
